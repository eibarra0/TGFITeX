% we include the glossary here (frontmatter is included with \input, so this command is as if it was in main.tex)
\input{frontmatter/glossary}

\frontmatter % Use roman page numbering style (i, ii, iii, iv...) for the pre-content pages

\pagestyle{thesis}

%----------------------------------------------------------------------------------------
%	TITLE PAGE
%----------------------------------------------------------------------------------------

\maketitlepage

%----------------------------------------------------------------------------------------
%	COUNCIL PAGE
%----------------------------------------------------------------------------------------

\begin{council}
{\setlength{\baselineskip}{0.5em}\setlength{\parindent}{1.5em}

{\raggedright{\textbf{\titcouncillabel}}\par}
\smallskip

Prof. Name1 Name2 Surname1 Surname2 \par
Prof. Name1 Name2 Surname1 Surname2 \par
Prof. Name1 Name2 Surname1 Surname2 \par
Prof. Name1 Name2 Surname1 Surname2 \par
Prof. Name1 Name2 Surname1 Surname2 \par
Prof. Name1 Name2 Surname1 Surname2 \par
Prof. Name1 Name2 Surname1 Surname2 \par
Prof. Name1 Name2 Surname1 Surname2 \par
Prof. Name1 Name2 Surname1 Surname2 \par

\null\vskip 10pt
	
{\raggedright{\textbf{\subscouncillabel}}\par}
\smallskip

Prof. Name1 Name2 Surname1 Surname2 \par
Prof. Name1 Name2 Surname1 Surname2 \par
Prof. Name1 Name2 Surname1 Surname2 \par
Prof. Name1 Name2 Surname1 Surname2 \par
Prof. Name1 Name2 Surname1 Surname2 \par
Prof. Name1 Name2 Surname1 Surname2 \par
Prof. Name1 Name2 Surname1 Surname2 \par
Prof. Name1 Name2 Surname1 Surname2 \par
Prof. Name1 Name2 Surname1 Surname2 \par

}
\end{council}



%----------------------------------------------------------------------------------------
%	DEDICATION  (optional)
%----------------------------------------------------------------------------------------


\begin{dedicatory}
\itshape
The dedicatory is optional. Below is an example of a humorous dedication.

\enquote{To my wife Marganit and my children Ella Rose and Daniel Adam without whom this book would have been completed two years earlier.} in \enquote{An Introduction To Algebraic Topology} by Joseph J. Rotman.

\end{dedicatory}

%----------------------------------------------------------------------------------------
%	ABSTRACT PAGE
%----------------------------------------------------------------------------------------

\begin{abstract}

The Thesis Abstract is written here (and usually kept to just this page). Abstract, is a term that is used to make a synthesis of some information, which because it is a fairly broad content, costs a little to process it completely. In this sense, it seeks to {\bfseries specify the most important points in order to be able to effectively process the information} \ldots

Please define up to 6 keywords that better describe your work, in the \emph{THESIS INFORMATION} block of the \file{main.tex} file.

\end{abstract}

\begin{abstractotherlanguage}

El Resumen del trabajo debe ser escrito aquí (y usualmente limitarse a esta página). Resumen, es un término que se emplea para realizar una síntesis de alguna información, que por ser un contenido bastante amplio, cuesta un poco procesarla por completo. En este sentido, se busca concretar los puntos mas importantes con el fin de poder {\bfseries procesar de forma efectiva la información} \ldots

Para modificar la lengua ir al archivo \file{main.tex} y cambiar a \enquote{english} o \enquote{spanish}. \footnote{Tener en cuenta revisar los macros implementados y la interacción entre diversos paquetes cargados.}. Realizar esta acción cambia automáticamente las referencias de capítulos para el contenido principal y en el preámbulo.

\end{abstractotherlanguage}

%----------------------------------------------------------------------------------------
%	ACKNOWLEDGEMENTS (optional)
%----------------------------------------------------------------------------------------

\begin{acknowledgements}

The optional Acknowledgment goes here\ldots Below is an example of a humorous acknowledgment.

\enquote{I'd also like to thank the Van Allen belts for protecting us from the harmful solar wind, and the earth for being just the right distance from the sun for being conducive to life, and for the ability for water atoms to clump so efficiently, for pretty much the same reason. Finally, I'd like to thank every single one of my forebears for surviving long enough in this hostile world to procreate. Without any one of you, this book would not have been possible.} in \enquote{The Woman Who Died a Lot} by Jasper Forde.

\end{acknowledgements}

%----------------------------------------------------------------------------------------
%	LIST OF CONTENTS/FIGURES/TABLES PAGES
%----------------------------------------------------------------------------------------

\tableofcontents % Prints the main table of contents

\listoffigures % Prints the list of figures

\listoftables % Prints the list of tables

\iflanguage{spanish}{
\renewcommand{\listalgorithmname}{Lista de Algorítmos}
}
\listofalgorithms % Prints the list of algorithms
\addchaptertocentry{\listalgorithmname}


\renewcommand{\lstlistlistingname}{List of Source Code}
\iflanguage{spanish}{
\renewcommand{\lstlistlistingname}{Lista de Códigos}
}
\lstlistoflistings % Prints the list of listings (programming language source code)
\addchaptertocentry{\lstlistlistingname}


%----------------------------------------------------------------------------------------
%	ABBREVIATIONS
%----------------------------------------------------------------------------------------
%\begin{abbreviations}{ll} % Include a list of abbreviations (a table of two columns)
%%\textbf{LAH} & \textbf{L}ist \textbf{A}bbreviations \textbf{H}ere\\
%%\textbf{WSF} & \textbf{W}hat (it) \textbf{S}tands \textbf{F}or\\
%\end{abbreviations}

%----------------------------------------------------------------------------------------
%	SYMBOLS
%----------------------------------------------------------------------------------------

\begin{symbols}{lll} % Include a list of Symbols (a three column table)

$a$ & distance & \si{\meter} \\
$P$ & power & \si{\watt} (\si{\joule\per\second}) \\
%Symbol & Name & Unit \\

\addlinespace % Gap to separate the Roman symbols from the Greek

$\omega$ & angular frequency & \si{\radian} \\

\end{symbols}



%----------------------------------------------------------------------------------------
%	ACRONYMS
%----------------------------------------------------------------------------------------

\newcommand{\listacronymname}{List of Acronyms}
\iflanguage{spanish}{
\renewcommand{\listacronymname}{Lista de Acrónimos}
}

%Use GLS
\glsresetall
\printglossary[title=\listacronymname,type=\acronymtype,style=long]

%----------------------------------------------------------------------------------------
%	DONE
%----------------------------------------------------------------------------------------

\mainmatter % Begin numeric (1,2,3...) page numbering
\pagestyle{thesis} % Return the page headers back to the "thesis" style
